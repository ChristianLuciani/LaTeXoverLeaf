\documentclass[]{article}
%package
\usepackage{siunitx}
\usepackage{mathtools}
\usepackage{xcolor}
\usepackage{fancybox}
\usepackage{tikz-cd}
\usepackage{tikz}
\usepackage{enumitem}
%\usetikzlibrary{matrix,decorations.pathmorphing,arrows}
%opening
\title{Modelo de Interaci\'on entre Poblaci\'on Migrante y Local}
\author{Antonio Di Teodoro; Georgina et al}

\begin{document}

\maketitle

\begin{abstract}
Esta es una aplicaci\'on de la metodolog\'ia del modelo de Islas y transferencia del conocimiento dentro del marco de clasificaci\'on e intercambio de objetos y dominios no medibles entre la población migrante y la poblaci\'on local . Este modelo se estudia en el contexto especifico de la poblaci\'on  de migrantes paraguayos asentados en la poblaci\'on en el Gran Rosario, Argentina, entre los años ???? y ????.
\end{abstract}

\section{Introducci\'on}

\section{Visi\'on Global del M\'etodo}

\textcolor{red}{Introducci\'on y justificaci\'on del m\'etodo ...}

\begin{itemize}
	\item \emph{Objetos  Organizaci\'on  Jerarquica.}   \ \  $\xi_{j}$ Para alg\'un   ${j \in N}$
	\item \emph{Espacio de  Intercambio (E.I.)}  \hspace{1cm} \begin{math} \Omega_{i j} ( \xi_{i}, \xi_{j}, f_{i j}, t_{i j}) \hspace{0.5cm} \forall i , j \in N \end{math}
	\item Comunidad
	\begin{equation}
	\Gamma_k = \bigcup^{k}_{i j \in \{1,\ldots,n\}}\Omega_{i j}
	\end{equation}
	donde  $ k \in \{1,\ldots,n\}$ representa el n\'umero de \emph{E I} a unir.
	\item Si  consideramos $ \hspace{0.5cm} h_I$ el conjunto de hip\'otesis internas, entonces definimos $\Omega_{i j} \cup h_{I_{i j}}$ al espacio de intercambio con h\'ipotesis internas. As\'i la \underline{Comunidad Interna}
	\begin{equation}
	\Gamma_{KI}=\bigcup^{K}_{i \in \{1,\ldots,n\}} \Omega_{i j}\cup h_{I_{i j}}
	\end{equation}
	Si consideramos $h_\epsilon$ el conjunto de hip\'otesis externa entonce definimos  la comunidad  externa o metacomunidad a la unión de la comunidad interna con el conjunto de hip\'otesis externas; es decir;
	\begin{equation}
	C_\epsilon =\mathcal{M} = \Gamma_{KI}\cup h_\epsilon
	\end{equation}
	
\end{itemize}



\paragraph{Ejemplo}
$\xi_i$ es igual a la misma clase social $i$: ``Conjunto''. \"Aqu\'i  $\xi_i$ representa una hip\'otesis de entrada $\gamma$ de formulaci\'on del problema\".

El mapa esta dado por
\begin{eqnarray}
\xi_i & \overset{f_{i j }}{\longrightarrow} & \xi_i  \\
a \in \xi_{i} & \longrightarrow & f_{ii}\left(a_{i}\right)
\end{eqnarray}

as\'i $f_{ii}$ transforma  en s\'i misma la  clase social $i$. \\
\underline{Notaci\'on:}\hspace{0.2cm} $\displaystyle \xi_i(t_{\xi_{i}}) \longrightarrow \xi_i(t_{\xi_{i}})$ \hspace{0.5cm} $\displaystyle f_{ii}[t_{f_{ii}}]\left( x\right) $\hspace{0.2cm}  donde \hspace{0.2cm} $x\leftarrow\xi_i(t_{\xi_i})$

\subsection{Capa 1 del Modelo}
\textcolor{red}{Descripci\'on de Capas...}

\begin{equation}
\xi_i(t_{\xi_i}) \xrightarrow[ ]{f_{ij}\left[ t_{f_{ij}} \right]} \xi_j(t_{ij})
\end{equation}

\emph{E.I :} ($\xi_i(t_{\xi_i}), \xi_j(\xi_j), f_{ij}[t_{ij}], t_{ij}$)  donde 
$t_{ij}$ depende de $t_{\xi_i},  t_{\xi_j},  t_{f_{ij}}$

\Ovalbox{
	\addtolength{\linewidth}{\fboxsep}%
	\addtolength{\linewidth}{\fboxrule}%
	\begin{minipage}{\linewidth}
		\begin{eqnarray*}
			\boxed{\xi_i(t_{\xi_i})} & \xleftrightarrow[]{f_{ij}\left[t_{f_{ij}}\right] } &\boxed{\xi_j(t_{ij})} \\
			t_{ij}&  &\Omega_{ij}
		\end{eqnarray*}
	\end{minipage}
}

\paragraph{\underline{Observaci\'on:}}
\begin{enumerate}
	\item $\xi_i \cap \xi_j = \emptyset$ \hspace{0.5cm} para \hspace{0.5cm}  $i\neq j$
	\item Si existe la inversa a $f_{ij}[t_{f_{ij}}]$ la denotaremos por $f_{ji}[t_{f_{ji}}]$\\
	As\'i existe $f^{-1}_{ij}[t_{f_{ij}}]=f_{ji}[t_{f_{ji}}]$.
\end{enumerate}

\subsection{Capa 2}

Si hay un solo \emph{EI}  $\Omega_{ij}$; escribimos

\begin{equation*}
\Gamma_{ij}=\Omega_{ij}
\end{equation*}
Para $i=j$ entonces $\Gamma_{ii}=\Omega_{ii}$.
As\'i $\displaystyle \Omega_{ij}{\overset{F_{ij}}{\longrightarrow}}$ o si $i=j$ $\land$ $k=l$ $\displaystyle \Omega_{ii}{\overset{F_{ik}}{\longrightarrow}}\Omega_{kk}$.\\



\begin{center}
	
\hrulefill

En general\\

\begin{equation*}
\Gamma_{n_1} \xrightarrow{F_{k_1k_2}} \Gamma_{n_2}
\end{equation*}
donde $n_1 , n_2$ identifican la comunidad.\\
\hrulefill
\end{center}


\paragraph{Ejemplo}
Considere la  misma clase social  $i$ $\Omega_{ii}$ y la misma  clase social $j$ $ \Omega_{jj}$
\begin{center}
	\cornersize{5}
\Ovalbox{$\Omega_{ii}$} $\xrightarrow{F_{ii}}$ \Ovalbox{$\Omega_{jj}$}
\end{center}
\subsection{Capa 1 y Capa 2}

\begin{center}
\cornersize*{2cm}
\Ovalbox{\ovalbox{\ovalbox{$\xi_1$}$ \xrightarrow[\Omega_{1 2}=\Gamma_{1 2}]{f_{1 2}}$\ovalbox{$\xi_2$}}$\xrightarrow{\mathcal{F}_{1234}}$\ovalbox{\ovalbox{$\xi_3$}$ \xrightarrow[\Omega_{3 4}=\Gamma_{3 4}]{f_{3 4}}$\ovalbox{$\xi_4$}}}
\end{center}

	
Observaci\'on creo que no es necesario agregar otra capa. Veamos el siguiente ejemplo.\\
Considere $\Gamma_2=\Omega_{1 2} \cup \Omega_{3 4}$; $\Gamma_{1 2}=\Omega_{1 2}$

\begin{center}
	\Ovalbox{$\Gamma_2$\Ovalbox{
			\begin{tikzcd}
				\ovalbox{$\xi_1$} \arrow{r}{f_{12}}& \ovalbox{$\xi_2$}\\
						 \ovalbox{$\xi_3$} \arrow{r}{f_{34}}&\ovalbox{$\xi_4$}
		\end{tikzcd}
}$\xrightarrow{\mathcal{F}_{2 1 , 2}}$\Ovalbox{
			\ovalbox{$\xi_1$} $\xrightarrow{f_{12}}$ \ovalbox{$\xi_2$}}$\Gamma_{1,2}$}
\end{center}
finalmente podemos considerar
\begin{center}
\begin{tikzcd}[column sep=1pc]

	\ovalbox{$\xi_1$}\arrow{r}{f_{12}} &\ovalbox{ $\xi_2$}\arrow{d}[swap]{f_{23}}&  \arrow{l}[swap]{f_{24}}  \ovalbox{$\xi_4$}\\ & \arrow[dotted]{ur} \ovalbox{ $\xi_3$} &\\
	\ovalbox{$\xi_5$} \arrow{r}{f_{56}}&\ovalbox{$\xi_6$}
\end{tikzcd}
\end{center}
Considerando $\Omega_{11}$;  $\Omega_{12}$; $\Omega_{23}$;  $\Omega_{34}$  y  $\Omega_{56}$. \\
Si existe $f_{43}$ podemos definir $\Omega_{43}$
\begin{eqnarray*}
	\Gamma_3 & =& \Omega_{11}\cup \Omega{12} \cup \Omega_{23}\\
	\Gamma_2 &=& \Omega_{34} \cup \Omega_{56}
\end{eqnarray*}
\newpage 
\begin{center}
\Ovalbox{
\begin{tikzcd}
\ovalbox{$\xi_1$}\arrow[bend left]{r}{f_{11}}   & \ovalbox{$\xi_1$}\arrow[bend left]{d}{f_{12}}\\
&\ovalbox{$\xi_2$} \arrow[bend right]{d}[swap]{f_{23}}\\
\Gamma_3& \ovalbox{$\xi_3$}	
\end{tikzcd}	}
\begin{tikzcd}
.\arrow[bend left]{r}{\mathcal{F}_{23}}&.
\end{tikzcd}
%$\xrightarrow{\Gamma_{23}}$
\Ovalbox{
	\begin{tikzcd}
		\ovalbox{$\xi_3$}\arrow[bend left]{rd}{f_{34}}&\\
		\ovalbox{$\xi_5$} \arrow[bend left]{rd}[swap]{f_{56}}&\ovalbox{$\xi_4$}\\
		\Gamma_2& \ovalbox{$\xi_6$}
	\end{tikzcd}
}
\end{center}

\hrulefill 
\subsection{No Agonistico Asimetr\'ia:}
\subsubsection{Problema Construcci\'on}
Considere los siguientes espacios de intercambio: \\

\begin{eqnarray*}
		\xi_1(t_{\xi_1})&=& \text{Ayudante set}\\
		\xi_2(t_{\xi_2})&=& \text{Medio oficial}\\
		\xi_3(t_{\xi_3})&=& \text{Oficial Especializado}\\
		\xi_4(t_{\xi_4})&=& \text{Contratista}\\
\end{eqnarray*}
\\
\emph{\underline{\text{CASO 1:}}}\\
Si la \textbf{ \textit{misma condici\'on megratoria} )mcs)} y la \textbf{ \textit{misma condici\'on socioecon\'omica} (mcs)} son objetos de organizaci\'on jerarquica entonces
\begin{eqnarray*}
	\xi_5(t_{\xi_5})&=&\text{mcm}\\
	\xi_5^c(t_{\xi_5})&=&\text{No-mcm}\\
	\xi_6(t_{\xi_6})&=&\text{mcs}\\
	\xi_6^c(t_{\xi_6})&=&\text{No-mcs}\\
\end{eqnarray*} 
\emph{Definimos un espacion de intercambio:}\\
\begin{equation}
	\Omega_{ij} = \left( \xi_i (t_{\xi_i} ), \xi_j (t_{\xi_j}), f_{ij}(t_{f_{ij}}), t_{ij} \right) \hspace{0.5cm} \forall i, j \in \{1,\dots,6\}
\end{equation}
Por ejemplo el caso ayudante y $\xi_k(t_k)=\xi_k$ con $t_{\xi_k}\rightarrow \infty$
\begin{equation}
	\Omega_{11}\xrightarrow{F_{11}[t_{f_{11}}]}\Omega_{11}
\end{equation}
El espacio de intercambio \textit{EI} $\Omega_{11} (\xi_1, \xi_1, f[t_{f_{11}}], t_11)$. La comunidad del ayudantes con \textbf{mcm} esto dado por 
\begin{equation}
	\Omega_{11} \cup \Omega_{55} \cup \Omega_15 = \Gamma_{15}
\end{equation} 
\begin{center}
\Ovalbox{
\begin{tikzcd}
	&\xi_1\arrow[bend left]{r}{}&\xi_5 \arrow[bend left]{dr}\\
	\xi_1\arrow[bend left]{ur}&&&\xi_5
\end{tikzcd}}
\end{center}

donde 
\begin{eqnarray}
	\Omega_{11}&=& \left(  \xi_1(t_{\xi_1}), \xi_1(t_{\xi_1}), f_{11}( t_{11}), t_{11} \right)\\
	\Omega_{55}&=& \left(  \xi_5(t_{\xi_5}), \xi_5(t_{\xi_5}), f_{55}(t_{55}), t_{55} \right)\\
	\Omega_{15}&=& \left(  \xi_1(t_{\xi_1}), \xi_5(t_{\xi_5}), f_{15}(t_{15}), t_{15} \right)
\end{eqnarray}
As\'i podemos establecer intercambios 
\begin{equation}
	\Gamma_1 \xrightarrow{{\mathcal{\tilde{F}}_{11}}}\Gamma_1
\end{equation}
Note que con nuestro modelo se puede establecer intercambios

\begin{itemize}[label=$\mapsto$]
\item \text{Ayudante} $\longrightarrow$ \text{Ayudante} \\
\item  \text{\textbf{mcm}} $\longrightarrow$ \text{\textbf{mcm}}\\
\item  \text{ Ayudante  y \textbf{mcm}} $\longrightarrow$  \text{Ayudante y/o \textbf{mcm}}\\
\item  \text{ Ayudante  \'o \textbf{mcm}}$\longrightarrow$ \text{ Ayudante y/o \textbf{mcm}}
\end{itemize}
Note que no es lo mismo $\Omega_{15}$ que $\Omega_{51}$ \\
Usando composici\'on podemos definir los cambios de estados como
\begin{equation}
\begin{tikzcd}
\Omega_{11} \arrow[bend right]{rrr}[swap]{\mathcal{F}_{1144}} \arrow{r}{\mathcal{F}_{1122}}
&\Omega_{22}\arrow{r}{\mathcal{F}_{2233}}
&\Omega_{33} \arrow{r}{\mathcal{F}_{3344}}
&\Omega_{44}
\end{tikzcd}
\end{equation}
Ojo si $\mathcal{F}_{1122}$ es un intercambio directo entonce $\mathcal{F}_{2211}$ es un intercambio inverso
\begin{equation}
	\begin{tikzcd}
	\Omega_{11}\arrow[bend left]{r}{\mathcal{F}_{1122}}\arrow{r}&\Omega_{22}\arrow[bend left]{l}{\mathcal{F}_{2211}}
	\end{tikzcd}
\end{equation} 
mapa -(ayudante-oficial )
\begin{eqnarray*}
	\Omega_{11}&\longrightarrow&\text{Transforma}\longrightarrow\Omega_{22}\\
	\text{Ayudante}&\longrightarrow&\mathcal{F}_{1122}\text{(Ayudante)}=\text{medio-oficial}
\end{eqnarray*}
Ls comunidad con estatus est\'a dada por 
\begin{equation}
\Gamma_2=\Omega_{11}\cup \Omega_{22} \cup \Omega_{33} \cup \Omega_{44} \cup \Omega_{55} \cup \Omega_{66}\cup \bigcup_{i<j} \Omega_{ij} \hspace{0.3cm}\text{con}\hspace{0.3cm} i,j \in \{1,\dots,6\}
\end{equation}
Se puede definir una comunidad de \textbf{ayudantes} m\'as \textbf{medio oficial} m\'as \textbf{mcm} m\'as \textbf{mcs}
\begin{equation}
\Gamma_3=\Omega_{11}\cup\Omega_{21}\cup\Omega_{55}\cup\Omega_{66}
\end{equation}
y podemos establecer intercambios 
\begin{equation*}
\Gamma_2\xrightarrow{\tilde{\mathcal{F}}_{23}}\Gamma_3
\end{equation*}
Observe 	que 
\begin{eqnarray*}
\text{si }\hspace{0.3cm}\Omega_{11}&\longrightarrow&\Omega_{22}\hspace{0.3cm} \text{ usamos } \hspace{0.3cm}\mathcal{F}_{12}\\
\text{si}\hspace{0.6cm}\Gamma_1&\longrightarrow&\Gamma_{3}\hspace{0.5cm} \text{ usamos }\hspace{0.3cm} \tilde{\mathcal{F}}_{13}
\end{eqnarray*}
\emph{\underline{\text{CASO 2:}}}\\
Si la misma condici\'on migratoria (\textbf{mcm}) y la misma condici\'on socio-econ\'omica (\textbf{mcs}) son ahora \emph{hip\'otesis internas} entoces:\\
\emph{Ejemplo}\\
Considere $h_1=$\textbf{mcm};  $h_2=$\textbf{mcs} y  los espacios de intercambios $\Omega_{11}$ (ayudantes ) y $\Omega_{22}$ (medio-oficial) as\'i
\begin{eqnarray}
	\Gamma_{1h_1}&=&\Omega_{11} \cup h_1 =\text{ayudantes y \textbf{mcm}} \\
	\Gamma_{2h2}&=&\Omega_{22} \cup h_2 =\text{medio  oficial y \textbf{mcs} }    
\end{eqnarray}  
As\'i definimos los intercambios entre las dos comunidades internas
\begin{equation}
 	\Gamma_{1h1}\xrightarrow{\tilde{\mathcal{F}}_{12}} \Gamma_{2 h_2}
\end{equation}
As\'i existe un intercambio entre ayudante m\'as \textbf{mcm} y medio oficial m\'as \textbf{mcs}. Note que este caso permite una manipulaci\'on distinta de \textbf{mcm} y \textbf{mcs} por lo que permita restructurar el el modelo seg\'un el estudio que se este realizando. Note que \textbf{mcm} y \textbf{mcs} aunque en este ejemplo son hip\'otesis deformativas. \\
 
Un ejemplo de hip\'otesis din\'amica puede ocurrir si :\\
\underline{Consideramos}:
\begin{equation}
\Omega_{11}\xrightarrow{\mathcal{F}_{11}} \Omega_{11}
\end{equation}
\begin{equation*}
\begin{tikzcd}
\ovalbox{$\xi_1$}\arrow{rr}{f_{11}}&\arrow{d}{\mathcal{F}_{11}}&\ovalbox{$\xi_1$}&\Omega_{11}\\
\ovalbox{$\xi_1$}\arrow{rr}[swap]{f_{11}}&{}&\ovalbox{$\xi_1$}&\Omega_{11}
\end{tikzcd}
\end{equation*}
\begin{eqnarray*}
\text{Considera}\hspace{0.5cm}  G_{1}&=&\text{Grupo de ayudantes} 1\\
 								  G_{2}&=&\text{Grupo de ayudantes} 2\\
 								  f_{11}&=&\begin{cases}
 								  	G_1\text{si estamos trabajando}\\
 								  	G_2\text{en el descanso}
 								  \end{cases}
\end{eqnarray*}
Note que $f_{11}$ no deforma $\Omega_{11}$. A diferencia de considerar sobre todos los ayudantes la misma condici\'on migratoria por ejemplo. 
\end{document}