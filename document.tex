\documentclass[]{article}
%package
\usepackage{siunitx}
\usepackage{mathtools}
\usepackage{xcolor}
\usepackage{fancybox}
\usepackage{tikz-cd}
\usepackage{tikz}
%\usetikzlibrary{matrix,decorations.pathmorphing,arrows}
%opening
\title{Modelo de Interaci\'on entre Poblaci\'on Migrante y Local}
\author{Antonio Di Teodoro; Georgina et al}

\begin{document}

\maketitle

\begin{abstract}
Esta es una aplicaci\'on de la metodolog\'ia del modelo de Islas y transferencia del conocimiento dentro del marco de clasificaci\'on e intercambio de objetos y dominios no medibles entre la población migrante y la poblaci\'on local . Este modelo se estudia en el contexto especifico de la poblaci\'on  de migrantes paraguayos asentados en la poblaci\'on en el Gran Rosario, Argentina, entre los años ???? y ????.
\end{abstract}

\section{Introducci\'on}

\section{Visi\'on Global del M\'etodo}

\textcolor{red}{Introducci\'on y justificaci\'on del m\'etodo ...}

\begin{itemize}
	\item \emph{O.  O.   I.}   \ \  $\xi_{j}$ Para alg\'un   ${j \in N}$
	\item \emph{E.  I.}  \hspace{1cm} \begin{math} \Omega_{i j} ( \xi_{i}, \xi_{j}, f_{i j}, t_{i j}) \hspace{0.5cm} \forall i , j \in N \end{math}
	\item Comunidad
	\begin{equation}
	\Gamma_k = \bigcup^{k}_{i j \in \{1,\ldots,n\}}\Omega_{i j}
	\end{equation}
	donde  $ k \in \{1,\ldots,n\}$ representa el n\'umero de \emph{E I} a unir.
	\item Si  consideramos $ \hspace{0.5cm} h_I$ el conjunto de hip\'otesis internas, entonces definimos $\Omega_{i j} \cup h_{I_{i j}}$ al espacio de intercambio con h\'ipotesis internas. As\'i la \underline{Comunidad Interna}
	\begin{equation}
	\Gamma_{KI}=\bigcup^{K}_{i \in \{1,\ldots,n\}} \Omega_{i j}\cup h_{I_{i j}}
	\end{equation}
	Si consideramos $h_\epsilon$ el conjunto de hip\'otesis externa entonce definimos  la comunidad  externa o metacomunidad a la unión de la comunidad interna con el conjunto de hip\'otesis externas; es decir;
	\begin{equation}
	C_\epsilon =\mathcal{M} = \Gamma_{KI}\cup h_\epsilon
	\end{equation}
	
\end{itemize}



\paragraph{Ejemplo}
$\xi_i$ es igual a la misma clase social $i$: ``Conjunto''. \"Aqu\'i  $\xi_i$ representa una hip\'otesis de entrada $\gamma$ de formulaci\'on del problema\".

El mapa esta dado por
\begin{eqnarray}
\xi_i & \overset{f_{i j }}{\longrightarrow} & \xi_i  \\
a \in \xi_{i} & \longrightarrow & f_{ii}\left(a_{i}\right)
\end{eqnarray}

as\'i $f_{ii}$ transforma  en s\'i misma la  clase social $i$. \\
\underline{Notaci\'on:}\hspace{0.2cm} $\displaystyle \xi_i(t_{\xi_{i}}) \longrightarrow \xi_i(t_{\xi_{i}})$ \hspace{0.5cm} $\displaystyle f_{ii}[t_{f_{ii}}]\left( x\right) $\hspace{0.2cm}  donde \hspace{0.2cm} $x\leftarrow\xi_i(t_{\xi_i})$

\subsection{Capa 1 del Modelo}
\textcolor{red}{Descripci\'on de Capas...}

\begin{equation}
\xi_i(t_{\xi_i}) \xrightarrow[ ]{f_{ij}\left[ t_{f_{ij}} \right]} \xi_j(t_{ij})
\end{equation}

\emph{E.I :} ($\xi_i(t_{\xi_i}), \xi_j(\xi_j), f_{ij}[t_{ij}], t_{ij}$)  donde 
$t_{ij}$ depende de $t_{\xi_i},  t_{\xi_j},  t_{f_{ij}}$

\Ovalbox{
	\addtolength{\linewidth}{\fboxsep}%
	\addtolength{\linewidth}{\fboxrule}%
	\begin{minipage}{\linewidth}
		\begin{eqnarray*}
			\boxed{\xi_i(t_{\xi_i})} & \xleftrightarrow[]{f_{ij}\left[t_{f_{ij}}\right] } &\boxed{\xi_j(t_{ij})} \\
			t_{ij}&  &\Omega_{ij}
		\end{eqnarray*}
	\end{minipage}
}

\paragraph{\underline{Observaci\'on:}}
\begin{enumerate}
	\item $\xi_i \cap \xi_j = \emptyset$ \hspace{0.5cm} para \hspace{0.5cm}  $i\neq j$
	\item Si existe la inversa a $f_{ij}[t_{f_{ij}}]$ la denotaremos por $f_{ji}[t_{f_{ji}}]$\\
	As\'i existe $f^{-1}_{ij}[t_{f_{ij}}]=f_{ji}[t_{f_{ji}}]$.
\end{enumerate}

\subsection{Capa 2}

Si hay un solo \emph{EI}  $\Omega_{ij}$; escribimos

\begin{equation*}
\Gamma_{ij}=\Omega_{ij}
\end{equation*}
Para $i=j$ entonces $\Gamma_{ii}=\Omega_{ii}$.
As\'i $\displaystyle \Omega_{ij}{\overset{F_{ij}}{\longrightarrow}}$ o si $i=j$ $\land$ $k=l$ $\displaystyle \Omega_{ii}{\overset{F_{ik}}{\longrightarrow}}\Omega_{kk}$.\\



\begin{center}
	
\hrulefill

En general\\

\begin{equation*}
\Gamma_{n_1} \xrightarrow{F_{k_1k_2}} \Gamma_{n_2}
\end{equation*}
donde $n_1 , n_2$ identifican la comunidad.\\
\hrulefill
\end{center}


\paragraph{Ejemplo}
Considere la  misma clase social  $i$ $\Omega_{ii}$ y la misma  clase social $j$ $ \Omega_{jj}$
\begin{center}
	\cornersize{5}
\Ovalbox{$\Omega_{ii}$} $\xrightarrow{F_{ii}}$ \Ovalbox{$\Omega_{jj}$}
\end{center}
\subsection{Capa 1 y Capa 2}

\begin{center}
\cornersize*{2cm}
\Ovalbox{\ovalbox{\ovalbox{$\xi_1$}$ \xrightarrow[\Omega_{1 2}=\Gamma_{1 2}]{f_{1 2}}$\ovalbox{$\xi_2$}}$\xrightarrow{F_{1234}}$\ovalbox{\ovalbox{$\xi_3$}$ \xrightarrow[\Omega_{3 4}=\Gamma_{3 4}]{f_{3 4}}$\ovalbox{$\xi_4$}}}
\end{center}

	
Observaci\'on creo que no es necesario agregar otra capa. Veamos el siguiente ejemplo.\\
Considere $\Gamma_2=\Omega_{1 2} \cup \Omega_{3 4}$; $\Gamma_{1 2}=\Omega_{1 2}$

\begin{center}
	\Ovalbox{$\Gamma_2$\Ovalbox{
			\ovalbox{$\xi_1$} $\xrightarrow{f_{12}}$ \ovalbox{$\xi_2$}\newline \ovalbox{$\xi_3$} $\xrightarrow{f_{34}}$ \ovalbox{$\xi_4$}}$\xrightarrow{F_{2 1 , 2}}$\Ovalbox{
			\ovalbox{$\xi_1$} $\xrightarrow{f_{12}}$ \ovalbox{$\xi_2$}}$\Gamma_{1,2}$}
\end{center}
finalmente podemos considerar
\begin{center}
\begin{tikzcd}[column sep=1pc]

	\ovalbox{$\xi_1$}\arrow{r}{f_{12}} &\ovalbox{ $\xi_2$}\arrow{d}[swap]{f_{23}}&  \arrow{l}[swap]{f_{24}}  \ovalbox{$\xi_4$}\\ & \arrow[dotted]{ur} \ovalbox{ $\xi_3$} &\\
	\ovalbox{$\xi_5$} \arrow{r}{f_{56}}&\ovalbox{$\xi_6$}
\end{tikzcd}
\end{center}
Considerando $\Omega_{11}$;  $\Omega_{12}$; $\Omega_{23}$;  $\Omega_{34}$  y  $\Omega_{56}$. \\
Si existe $f_{43}$ podemos definir $\Omega_{43}$
\begin{eqnarray*}
	\Gamma_3 & =& \Omega_{11}\cup \Omega{12} \cup \Omega_{23}\\
	\Gamma_2 &=& \Omega_{34} \cup \Omega_{56}
\end{eqnarray*}

\begin{tikzcd}
	\Ovalbox{\ovalbox{$\xi_1$}}}
\end{tikzcd}	
\end{document}